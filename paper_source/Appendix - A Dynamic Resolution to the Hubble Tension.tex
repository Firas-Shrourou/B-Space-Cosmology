%%%%%%%%%%%%%%%%%%%%%%%%%%%%%%%%%%%%%%%%%%%%%%%%%%%%%%%%%%%%%%%%%%%%
% Appendix A: A Dynamic Resolution to the Hubble Tension
% This file now uses the BSpacePaper.cls class for all formatting.
%%%%%%%%%%%%%%%%%%%%%%%%%%%%%%%%%%%%%%%%%%%%%%%%%%%%%%%%%%%%%%%%%%%%

\documentclass{BSpacePaper} % <-- THE MAGIC HAPPENS HERE!

% --- METADATA FOR THIS SPECIFIC PAPER ---
\papertitle{A Dynamic Resolution to the Hubble Tension in B-Space Cosmology}
\paperauthor{Firas Shrourou \\ \href{https://github.com/Firas-Shrourou/B-Space-Cosmology}{github.com/Firas-Shrourou/B-Space-Cosmology}}
\papersubject{A rigorous proof demonstrating how the B-Space cosmological model's volumetric drag mechanism naturally resolves the Hubble Tension by creating a time-dependent expansion history consistent with both early and late-universe observations.}
\paperkeywords{Cosmology, B-Space, Hubble Tension, Dark Energy, Drag, Lambda-CDM}

\begin{document}

\makeBSCSsupplementtitle

\begin{abstract}
\noindent
The persistent discrepancy between early-universe and late-universe measurements of the Hubble constant ($H_0$) is a foundational crisis in modern cosmology. This paper provides a rigorous, step-by-step proof of how the B-Space cosmological model naturally resolves this tension. We demonstrate that the model's core physical mechanism—a volumetric drag force that grows over cosmic time—was negligible during the post-recombination epoch, forcing the model to reproduce the low $H_0$ value inferred from CMB data. We then show how the subsequent growth of this drag force reshapes the entire cosmic expansion history ($H(z)$), leading to an accelerated expansion that matches the high $H_0$ value observed in the local universe today. This resolution is not a fine-tuning but a direct consequence of the model's data-constrained parameters.
\end{abstract}

\section{The Problem Defined: A Conflict of Prediction and Measurement}
The Hubble Tension is a conflict between two numbers:
\begin{itemize}
    \item \textbf{The Prediction from the Early Universe:} The standard \(\Lambda\)CDM model, when fit to the Cosmic Microwave Background (CMB) data from the Planck satellite ($z \approx 1100$), predicts a present-day expansion rate of \textbf{$H_0 = 67.4 \pm 0.5$} km/s/Mpc \citep{Planck2020}.
    \item \textbf{The Direct Measurement in the Late Universe:} Measurements of the local universe using Type Ia Supernovae calibrated with Cepheid variables (the SH0ES project) find an expansion rate of \textbf{$H_0 = 73.0 \pm 1.0$} km/s/Mpc \citep{Riess2022}.
\end{itemize}
These values disagree at a statistical significance of $\sim 5\sigma$. A successful theory must explain why a model based on the early universe implies a low value while the present-day universe demonstrably has a high value.

\section{The \bspace{} Mechanism: A Time-Evolving Brake}
The \bspace{} solution lies in its Master Equation, which contains a time-evolving drag term that grows with the scale factor, $a$:
\begin{equation}
    \frac{\ddot{a}}{a} = -\frac{4\pi G}{3}\rho_m + \frac{8\pi G}{3}\rho_{\text{ext}} - \left(\Gamma_0 a^3\right) H
\end{equation}
The key to resolving the tension is the behavior of the volumetric drag term, $(\Gamma_0 a^3)H$, at different cosmic epochs.

\section{Analysis of the Early Universe ($z \approx 1100$)}
At the time of recombination, the scale factor was tiny, $a = 1/(1+z) \approx 1/1101 \approx 9.1 \times 10^{-4}$. The drag term's strength is proportional to $a^3$:
\[ a^3 \approx (9.1 \times 10^{-4})^3 \approx 7.5 \times 10^{-10} \]
This means the drag force was \textbf{effectively zero} and had no impact on the physics of the early universe. The \bspace{} Master Equation at that time reduced to:
\begin{equation}
    \frac{\ddot{a}}{a} \bigg|_{z=1100} \approx -\frac{4\pi G}{3}\rho_m + \frac{8\pi G}{3}\rho_{\text{ext}}
\end{equation}
This equation is mathematically indistinguishable from the standard \lcdm{} equation. Therefore, when the \bspace{} model is fit to the Planck CMB data, it is \textbf{guaranteed to reproduce the same physics and the same resulting prediction} for the present-day Hubble constant as \lcdm{}. The model naturally explains the Planck result.

\section{Analysis of the Late Universe ($z < 2$)}
In the late universe, as $a$ has grown towards its present-day value of 1, the drag term has become a significant component of the cosmic energy budget. It acts as a continuous brake on the expansion, altering the shape of the expansion history $H(z)$ relative to the drag-free \lcdm{} model. The equation governing the expansion rate can be derived from the Master Equation:
\begin{equation}
    H^2(a) = H_0^2 \left[ \Omega_m a^{-3} + \Omega_{\text{ext}} - \frac{2\Gamma_0}{3H_0^2} \int_1^a a'^3 H(a') da' \right]
\end{equation}
The presence of the \textbf{drag integral} is the crucial difference from \lcdm{}. This term explicitly shows that the expansion rate today depends on the entire past history of the drag force.

\section{Conclusion: A Dynamic Resolution}
The Hubble Tension is not a paradox in \bspace{} Cosmology; it is a \textbf{direct measurement of the total integrated effect of cosmic drag.}
\begin{itemize}
    \item The model reproduces the Planck prediction because the drag was inactive in the early universe.
    \item The model matches the local measurement because the growth of the drag force over billions of years has altered the expansion dynamics.
\end{itemize}
The fundamental drag constant, $\Gamma_0$, is a parameter that is measured by the very existence of the tension. A full MCMC analysis is required to find its precise value, but the mechanism itself provides a complete, dynamic, and physically motivated resolution to the Hubble Tension.

\bibliographystyle{plainnat}
\begin{thebibliography}{9}

\bibitem[Planck Collaboration(2020)]{Planck2020}
N. Aghanim, et al. (Planck Collaboration). 2020,
\textit{Planck 2018 results. VI. Cosmological parameters},
A\&A, 641, A6.

\bibitem[Riess et al.(2022)]{Riess2022}
A. G. Riess, et al. 2022,
\textit{A Comprehensive Measurement of the Local Value of the Hubble Constant with 1 km/s/Mpc Uncertainty from the Hubble Space Telescope and the SH0ES Team},
ApJL, 934, L7.

\end{thebibliography}

\end{document}