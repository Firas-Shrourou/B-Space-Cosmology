%%%%%%%%%%%%%%%%%%%%%%%%%%%%%%%%%%%%%%%%%%%%%%%%%%%%%%%%%%%%%%%%%%%%
% Addressing Dark Matter in B-Space Cosmology
% This version uses the BSpacePaper.cls class for all formatting.
% Author: Firas Shrourou
% Date: July 2025
%%%%%%%%%%%%%%%%%%%%%%%%%%%%%%%%%%%%%%%%%%%%%%%%%%%%%%%%%%%%%%%%%%%%

\documentclass{BSpacePaper} % Use our master class file

% --- METADATA FOR THIS SPECIFIC PAPER ---
\papertitle{Addressing Dark Matter in B-Space Cosmology}
\paperauthor{Firas Shrourou}
\papersubject{A supplementary document detailing the re-contextualization of dark matter as a property of the B-Space background medium, rather than a particle component of the universe.}
\paperkeywords{Dark Matter, B-Space, Dark Matter Halos, WIMP, Axion, Cosmological Model}

\begin{document}

\makeBSCSsupplementtitle

\begin{abstract}
\noindent
In standard cosmology, dark matter (DM) is an unknown particle species comprising $\sim$26.5\% of the universe's energy density. B-Space Cosmology offers a radical alternative: dark matter is not a particle, but is the physical fabric of the static background medium our universe expands into. This paper details this redefinition. We show how the gravitational effects attributed to DM arise from the mechanical deformation of the B-Space fabric. This framework naturally resolves the search for DM particles by postulating that none exist within our universe, and it makes sharp, falsifiable predictions regarding the relationship between baryonic matter and DM halo profiles.
\end{abstract}

\section{Core Clarification: Dark Matter is Not in the Drip}
In B-Space Cosmology, the concepts of baryonic matter and dark matter are fundamentally separated.
\begin{itemize}
    \item The \textbf{Drip} (our observable universe) contains only baryonic matter and radiation. Its total mass, $M$, which appears in all gravitational and dynamical equations, is purely baryonic.
    \item The \textbf{B-Space} background is the physical manifestation of dark matter. Its mechanical properties (e.g., density, elasticity) and its deformation under the Drip's gravity generate the large-scale gravitational effects attributed to DM.
\end{itemize}
This distinction has immediate consequences: it resolves the ongoing "crisis" of the non-detection of WIMPs, axions, or other particle-DM candidates by positing that no such particles exist to be found within our universe.

\section{Reconciling the Cosmological Energy Budget}
The standard \(\Lambda\)CDM energy budget is reallocated in the B-Space framework. The 26.5\% energy fraction attributed to particle dark matter is not needed, as its gravitational effects are now explained by the B-Space medium itself.

% --- Using non-floating table for stability ---
\begin{center}
    \captionsetup{type=table}
    \captionof{table}{Reallocation of Cosmological Components}
    \label{tab:reallocation}
    \begin{tabular}{@{}lll@{}}
    \toprule
    \textbf{Component} & \textbf{\(\Lambda\)CDM Energy Fraction} & \textbf{B-Space Equivalent} \\
    \midrule
    Baryonic Matter & 4.9\% & Contained within the Drip ($M$) \\
    \addlinespace
    Particle Dark Matter & 26.5\% & Replaced by the physical B-Space fabric \\
    \addlinespace
    Dark Energy (\(\Lambda\)) & 68.6\% & Replaced by mechanical forces ($\Delta P$, Drag) \\
    \bottomrule
    \end{tabular}
\end{center}
The key insight is that B-Space's DM effects are non-energetic in the context of the Drip's mass budget. They emerge from the interaction with the background, not from mass-energy contained within the Drip itself.

\section{Decoupling Dark Matter from Inflation}
B-Space Cosmology fundamentally alters the origin story of dark matter.
\begin{itemize}
    \item \textbf{Inflation Affects the Drip Only:} Inflation and reheating are events that set the initial conditions of the Drip.
    \item \textbf{B-Space is Primordial:} B-Space is a pre-existing, atemporal manifold. Its "dark matter" properties are intrinsic and were not generated by the thermal history of the Drip.
    \item \textbf{No WIMP Freeze-Out:} This completely obviates the need for thermal freeze-out mechanisms and solves the associated "WIMP miracle" fine-tuning problem.
\end{itemize}

\section{Mathematical and Physical Consistency}
The B-Space framework remains mathematically consistent using only the baryonic mass $M$.
\begin{itemize}
    \item \textbf{Self-Gravity:} The term $\ddot{R}_{\text{gravity}}=-GM/R^2$ is correct because $M$ is the only source of gravity \textit{within} the Drip. The gravity-like effects of B-Space are accounted for separately, for instance, in the constant pressure term $\rho_{\text{ext}}$.
    \item \textbf{Galaxy Dynamics:} Observed dynamics like galaxy rotation curves are not caused by local DM particles. They are explained by the strain induced in the B-Space fabric by the presence of baryonic matter ($\varepsilon_{ij} \propto \nabla^2\Phi_{\text{baryonic}}$).
    \item \textbf{Thermodynamics:} The drag heating term $\dot{Q} \propto (\Gamma_0 a^3) \rho_m \dot{R}^2$ correctly uses the baryonic density $\rho_m$, as it is the baryonic matter that is being dragged through the B-Space medium.
\end{itemize}

\section{Falsifiable Distinctions from \(\Lambda\)CDM}
This redefinition of dark matter leads to sharp, testable predictions that differ fundamentally from standard particle DM models. If B-Space is valid:
\begin{enumerate}
    \item \textbf{No particle dark matter will ever be directly or indirectly detected.} All searches will yield null results.
    \item \textbf{Dark matter "halo" profiles must correlate uniquely with the baryonic mass distribution that creates them.} They should not perfectly match the profiles from collisionless N-body simulations of particle DM.
    \item \textbf{High-redshift galaxies ($z>6$) should exhibit weaker "DM-like" effects}, as the B-Space deformation effect is predicted to scale with the Drip's radius ($R^{-1}$), which was smaller in the past.
\end{enumerate}

\section{Conclusion}
B-Space Cosmology eliminates the need for particle dark matter by redefining it as the physical fabric of a static background. The model's equations correctly and consistently use only the baryonic mass of the universe. This framework is not only self-consistent but also highly falsifiable, predicting a definitive and permanent lack of particle DM detection and a unique relationship between baryonic matter and the gravitational effects attributed to dark matter.

\end{document}